
%-------------------------%
%-----Document Setup------%
%-------------------------%
\documentclass[12pt,oneside,openany]{memoir}
\usepackage[utf8x]{inputenc}
\usepackage[english]{babel}
\usepackage{savetrees}
\linespread{1.6}
\setlength{\footskip}{20pt}
\usepackage{ifthen}

\ifthenelse{\lengthtest { \paperwidth = 11in}}
    { \geometry{top=.5in,left=.5in,right=.5in,bottom=.5in} }
    {\ifthenelse{ \lengthtest{ \paperwidth = 297mm}}
        {\geometry{top=3cm,left=3cm,right=3cm,bottom=3cm} }
        {\geometry{top=2cm,left=3cm,right=3cm,bottom=2cm} }
    }

%-------------------------%
%------Document Code------%
%-------------------------%
\newcommand{\thought}[1]{\textit{#1}}

\newcommand{\scenechange}{
  \par
  \vspace{\baselineskip}
  \par
\noindent}
%Creates a line break for a change of scene

\newcommand{\majorchange}{
  \par
  \vspace{\baselineskip}
  \hfill
  \textasteriskcentered
  \hfill
  \vspace{\baselineskip}
\noindent}
%creates a major line break, split by an asterisk for scene changes at the end of a page of where a sense of a major change is required. 

%-------------------------%
%------Main Document------%
%-------------------------%
\begin{document}
\title{30,000 Feet}
\author{Noah Eisen}
\date{}
\maketitle
\thispagestyle{empty}
\pagestyle{empty}
\setlength{\parindent}{30pt}
%\section*{}

Roughly 30,000 feet above a bucolic stretch of land connecting Chicago and San Francisco, a baby died on an airplane. Nate had been sitting one row back, angry at the baby for its shrill wails, then the baby had died and Nate sat there blinking as he listened to the shrill wails of the mother. 

Nate did not know the baby personally, nor did he know the mother. They were strangers. He did not know that the name of the baby had been Rachel, and he did not know that the baby had been born with a weak heart, which was the cause of its untimely death. 

He did know that the baby had been sitting exactly one row in front of him and it had died early in the flight from Chicago to San Francisco after wailing for exactly one hour. 

He and fifty others were an unwilling audience to the anguish of the mother as the flight attendants gave her a blanket with which to wrap the dead baby. Nate imagined that a terse conversation must had occurred between the pilots and ground control, for the logical yet seemingly inhuman decision was made to continue all the way to the final destination. The rest of the trip passed in silence.

“How was the flight?” his wife asked him when he returned home.

“A baby sitting in front of me died.”

“Oh.”

	Nate could not get to sleep that night. His head echoed with the shrill wails of the dead baby, and then the shrill wails of the childless mother, and then both together in a cacophonous harmony. This continued the next night.
	
	“I think you need to see someone.” His wife finally insisted. After a month she had realized that her husband had been deeply affected by the tragedy, and that his sallow and empty face wouldn't repair itself without help. Nate knew she was right.
	
	“So, tell me what brings you here.” Dr. Aaron sat back in her chair as she posed the question. After years of practicing psychology she believed that she had developed a talent to see sadness behind peoples eyes, and Nate was full to the brim.
	
	“During a flight last month a baby sitting in front of me wailed for an hour and then died.”
	
	“Oh.”
	
	The therapy continued. It might have helped. After two months Dr. Aaron posed a suggestion.
	
	Nate and his wife sat at park by their house for hours. The park was full of parents and babies making noises. Nate was happy with the suggestion that Dr. Aaron had given him.
	
	Slowly, Nate was recovering. Very slowly. He still had many sleepless nights as he rolled in bed, tormented by the shrill wails. Some nights he became very angry with the baby for dying, and for dying precisely when and where it did. Some nights he became very angry with himself for having been on that particular flight, and for sitting exactly where he did. Without exception, these thoughts would then lead him to become angry with fate and chance and the incomprehensibly mundane nature of tragedy.
	
	Every Saturday afternoon, Nate and his wife would return to the park and sit for hours and listen to the noises of the babies and their parents. Usually the noises were a mix of laughter and giggling and yells, but sometimes the noises included shrill wails and Nate would clench his hands.
	
	Eventually his sadness diminished, but something else grew in its place. The something was not as much a something, as it was a lack of something. He found within himself a profound emptiness that he had not ever expected to find. He now spent his sleepless nights pondering this emptiness, wondering what had caused this black hole to open up inside of him, and how in the name of God had he not sensed it before the baby died on the plane.
	
	After a month of pondering, Nate had an idea. He first told Dr. Aaron and she liked it. He then went to his wife and she was not at all surprised, for she, too, had sensed the emptiness within Nate, and had mused over what could remedy it.
	
One week after Nate’s 50th birthday he and his wife adopted a baby and named her Lila, and on occasion, as she grew older, she would have fits of shrill wails, and Nate cherished every single moment of them. 




\end{document}